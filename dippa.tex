% Lines starting with a percent sign (%) are comments. LaTeX will
% not process those lines. Similarly, everything after a percent
% sign in a line is considered a comment. To produce a percent sign
% in the output, write \% (backslash followed by the percent sign).
% ==================================================================
% Usage instructions:
% ------------------------------------------------------------------
% The file is heavily commented so that you know what the various
% commands do. Feel free to remove any comments you don't need from
% your own copy. When redistributing the example thesis file, please
% retain all the comments for the benefit of other thesis writers!
% ==================================================================
% Compilation instructions:
% ------------------------------------------------------------------
% Use pdflatex to compile! Input images are expected as PDF files.
% Example compilation:
% ------------------------------------------------------------------
% > pdflatex thesis-example.tex
% > bibtex thesis-example
% > pdflatex thesis-example.tex
% > pdflatex thesis-example.tex
% ------------------------------------------------------------------
% You need to run pdflatex multiple times so that all the cross-references
% are fixed. pdflatex will tell you if you need to re-run it (a warning
% will be issued)
% ------------------------------------------------------------------
% Compilation has been tested to work in ukk.cs.hut.fi and kosh.hut.fi
% - if you have problems of missing .sty -files, then the local LaTeX
% environment does not have all the required packages installed.
% For example, when compiling in vipunen.hut.fi, you get an error that
% tikz.sty is missing - in this case you must either compile somewhere
% else, or you cannot use TikZ graphics in your thesis and must therefore
% remove or comment out the tikz package and all the tikz definitions.
% ------------------------------------------------------------------

% General information
% ==================================================================
% Package documentation:
%
% The comments often refer to package documentation. (Almost) all LaTeX
% packages have documentation accompanying them, so you can read the
% package documentation for further information. When a package 'xxx' is
% installed to your local LaTeX environment (the document compiles
% when you have \usepackage{xxx} and LaTeX does not complain), you can
% find the documentation somewhere in the local LaTeX texmf directory
% hierarchy. In ukk.cs.hut.fi, this is /usr/texlive/2008/texmf-dist,
% and the documentation for the titlesec package (for example) can be
% found at /usr/texlive/2008/texmf-dist/doc/latex/titlesec/titlesec.pdf.
% Most often the documentation is located as a PDF file in
% /usr/texlive/2008/texmf-dist/doc/latex/xxx, where xxx is the package name;
% however, documentation for TikZ is in
% /usr/texlive/2008/texmf-dist/doc/latex/generic/pgf/pgfmanual.pdf
% (this is because TikZ is a front-end for PGF, which is meant to be a
% generic portable graphics format for LaTeX).
% You can try to look for the package manual using the ``find'' shell
% command in Linux machines; the find databases are up-to-date at least
% in ukk.cs.hut.fi. Just type ``find xxx'', where xxx is the package
% name, and you should find a documentation file.
% Note that in some packages, the documentation is in the DVI file
% format. In this case, you can copy the DVI file to your home directory,
% and convert it to PDF with the dvipdfm command (or you can read the
% DVI file directly with a DVI viewer).
%
% If you can't find the documentation for a package, just try Googling
% for ``latex packagename''; most often you can get a direct link to the
% package manual in PDF format.
% ------------------------------------------------------------------


% Document class for the thesis is report
% ------------------------------------------------------------------
% You can change this but do so at your own risk - it may break other things.
% Note that the option pdftext is used for pdflatex; there is no
% pdflatex option.
% ------------------------------------------------------------------
\documentclass[12pt,a4paper,oneside,pdftex]{report}

% The input files (tex files) are encoded with the latin-1 encoding
% (ISO-8859-1 works). Change the latin1-option if you use UTF8
% (at some point LaTeX did not work with UTF8, but I'm not sure
% what the current situation is)
\usepackage[utf8]{inputenc}
% OT1 font encoding seems to work better than T1. Check the rendered
% PDF file to see if the fonts are encoded properly as vectors (instead
% of rendered bitmaps). You can do this by zooming very close to any letter
% - if the letter is shown pixelated, you should change this setting
% (try commenting out the entire line, for example).
\usepackage[OT1]{fontenc}
% The babel package provides hyphenating instructions for LaTeX. Give
% the languages you wish to use in your thesis as options to the babel
% package (as shown below). You can remove any language you are not
% going to use.
% Examples of valid language codes: english (or USenglish), british,
% finnish, swedish; and so on.
\usepackage[finnish,swedish,english]{babel}


% Font selection
% ------------------------------------------------------------------
% The default LaTeX font is a very good font for rendering your
% thesis. It is a very professional font, which will always be
% accepted.
% If you, however, wish to spicen up your thesis, you can try out
% these font variants by uncommenting one of the following lines
% (or by finding another font package). The fonts shown here are
% all fonts that you could use in your thesis (not too silly).
% Changing the font causes the layouts to shift a bit; you many
% need to manually adjust some layouts. Check the warning messages
% LaTeX gives you.
% ------------------------------------------------------------------
% To find another font, check out the font catalogue from
% http://www.tug.dk/FontCatalogue/mathfonts.html
% This link points to the list of fonts that support maths, but
% that's a fairly important point for master's theses.
% ------------------------------------------------------------------
% <rant>
% Remember, there is no excuse to use Comic Sans, ever, in any
% situation! (Well, maybe in speech bubbles in comics, but there
% are better options for those too)
% </rant>

% \usepackage{palatino}
% \usepackage{tgpagella}



% Optional packages
% ------------------------------------------------------------------
% Select those packages that you need for your thesis. You may delete
% or comment the rest.

% Natbib allows you to select the format of the bibliography references.
% The first example uses numbered citations:
%\usepackage[square,sort&compress,numbers]{natbib}
% The second example uses author-year citations.
% If you use author-year citations, change the bibliography style (below);
% acm style does not work with author-year citations.
% Also, you should use \citet (cite in text) when you wish to refer
% to the author directly (\citet{blaablaa} said blaa blaa), and
% \citep when you wish to refer similarly than with numbered citations
% (It has been said that blaa blaa~\citep{blaablaa}).
 \usepackage[round]{natbib}

% The alltt package provides an all-teletype environment that acts
% like verbatim but you can use LaTeX commands in it. Uncomment if
% you want to use this environment.
% \usepackage{alltt}

% The eurosym package provides a euro symbol. Use with \euro{}
\usepackage{eurosym}

% Verbatim provides a standard teletype environment that renderes
% the text exactly as written in the tex file. Useful for code
% snippets (although you can also use the listings package to get
% automatic code formatting).
\usepackage{verbatim}

% The listing package provides automatic code formatting utilities
% so that you can copy-paste code examples and have them rendered
% nicely. See the package documentation for details.
% \usepackage{listings}

% The fancuvrb package provides fancier verbatim environments
% (you can, for example, put borders around the verbatim text area
% and so on). See package for details.
% \usepackage{fancyvrb}

% Supertabular provides a tabular environment that can span multiple
% pages.
%\usepackage{supertabular}
% Longtable provides a tabular environment that can span multiple
% pages. This is used in the example acronyms file.
\usepackage{longtable}

% The fancyhdr package allows you to set your the page headers
% manually, and allows you to add separator lines and so on.
% Check the package documentation.
% \usepackage{fancyhdr}

% Subfigure package allows you to use subfigures (i.e. many subfigures
% within one figure environment). These can have different labels and
% they are numbered automatically. Check the package documentation.
\usepackage{subfigure}

% The titlesec package can be used to alter the look of the titles
% of sections, chapters, and so on. This example uses the ``medium''
% package option which sets the titles to a medium size, making them
% a bit smaller than what is the default. You can fine-tune the
% title fonts and sizes by using the package options. See the package
% documentation.
\usepackage[medium]{titlesec}

% The TikZ package allows you to create professional technical figures.
% The learning curve is quite steep, but it is definitely worth it if
% you wish to have really good-looking technical figures.
\usepackage{tikz}
% You also need to specify which TikZ libraries you use
\usetikzlibrary{positioning}
\usetikzlibrary{calc}
\usetikzlibrary{arrows}
\usetikzlibrary{decorations.pathmorphing,decorations.markings}
\usetikzlibrary{shapes}
\usetikzlibrary{patterns}


% The aalto-thesis package provides typesetting instructions for the
% standard master's thesis parts (abstracts, front page, and so on)
% Load this package second-to-last, just before the hyperref package.
% Options that you can use:
%   mydraft - renders the thesis in draft mode.
%             Do not use for the final version.
%   doublenumbering - [optional] number the first pages of the thesis
%                     with roman numerals (i, ii, iii, ...); and start
%                     arabic numbering (1, 2, 3, ...) only on the
%                     first page of the first chapter
%   twoinstructors  - changes the title of instructors to plural form
%   twosupervisors  - changes the title of supervisors to plural form
\usepackage[mydraft,twosupervisors]{aalto-thesis}
%\usepackage[mydraft,doublenumbering]{aalto-thesis}
%\usepackage{aalto-thesis}


% Hyperref
% ------------------------------------------------------------------
% Hyperref creates links from URLs, for references, and creates a
% TOC in the PDF file.
% This package must be the last one you include, because it has
% compatibility issues with many other packages and it fixes
% those issues when it is loaded.
\RequirePackage[pdftex]{hyperref}
% Setup hyperref so that links are clickable but do not look
% different
\hypersetup{colorlinks=false,raiselinks=false,breaklinks=true}
\hypersetup{pdfborder={0 0 0}}
\hypersetup{bookmarksnumbered=true}
% The following line suggests the PDF reader that it should show the
% first level of bookmarks opened in the hierarchical bookmark view.
\hypersetup{bookmarksopen=true,bookmarksopenlevel=1}
% Hyperref can also set up the PDF metadata fields. These are
% set a bit later on, after the thesis setup.


% Thesis setup
% ==================================================================
% Change these to fit your own thesis.
% \COMMAND always refers to the English version;
% \FCOMMAND refers to the Finnish version; and
% \SCOMMAND refers to the Swedish version.
% You may comment/remove those language variants that you do not use
% (but then you must not include the abstracts for that language)
% ------------------------------------------------------------------
% If you do not find the command for a text that is shown in the cover page or
% in the abstract texts, check the aalto-thesis.sty file and locate the text
% from there.
% All the texts are configured in language-specific blocks (lots of commands
% that look like this: \renewcommand{\ATCITY}{Espoo}.
% You can just fix the texts there. Just remember to check all the language
% variants you use (they are all there in the same place).
% ------------------------------------------------------------------
\newcommand{\TITLE}{Presales}
\newcommand{\FTITLE}{Presales-funktio}

\newcommand{\SUBTITLE}{How presales should be organized at Sievo? }
\newcommand{\FSUBTITLE}{Tapaus Sievo}
\newcommand{\DATE}{March 22, 2013}
\newcommand{\FDATE}{22. maaliskuuta 2013}
\newcommand{\ATDEGREEPROG}{Degree Programme in Information Networks}

% Supervisors and instructors
% ------------------------------------------------------------------
% If you have two supervisors, write both names here, separate them with a
% double-backslash (see below for an example)
% Also remember to add the package option ``twosupervisors'' or
% ``twoinstructors'' to the aalto-thesis package so that the titles are in
% plural.
% Example of one supervisor:
%\newcommand{\SUPERVISOR}{Professor Antti Ylä-Jääski}
%\newcommand{\FSUPERVISOR}{Professori Antti Ylä-Jääski}
%\newcommand{\SSUPERVISOR}{Professor Antti Ylä-Jääski}
% Example of twosupervisors:
\newcommand{\SUPERVISOR}{Professor Eila Järvenpää}
\newcommand{\FSUPERVISOR}{Professori Eila Järvenpää}


% If you have only one instructor, just write one name here
\newcommand{\INSTRUCTOR}{Sammeli Sammalkorpi M.Sc. (Tech.)}
\newcommand{\FINSTRUCTOR}{Diplomi-insinööri Sammeli Sammalkorpi}

% If you have two instructors, separate them with \\ to create linefeeds
% \newcommand{\INSTRUCTOR}{Olli Ohjaaja M.Sc. (Tech.)\\
%  Elli Opas M.Sc. (Tech)}
%\newcommand{\FINSTRUCTOR}{Diplomi-insinööri Olli Ohjaaja\\
%  Diplomi-insinööri Elli Opas}
%\newcommand{\SINSTRUCTOR}{Diplomingenjör Olli Ohjaaja\\
%  Diplomingenjör Elli Opas}

% If you have two supervisors, it is common to write the schools
% of the supervisors in the cover page. If the following command is defined,
% then the supervisor names shown here are printed in the cover page. Otherwise,
% the supervisor names defined above are used.
\newcommand{\COVERSUPERVISOR}{Professor Eila Järvenpää, Aalto University}

% The same option is for the instructors, if you have multiple instructors.
% \newcommand{\COVERINSTRUCTOR}{Olli Ohjaaja M.Sc. (Tech.), Aalto University\\
%  Elli Opas M.Sc. (Tech), Aalto SCI}


% Other stuff
% ------------------------------------------------------------------
\newcommand{\PROFESSORSHIP}{Knowldge Intensive Business}
\newcommand{\FPROFESSORSHIP}{Tietointensiivinen liiketoiminta}

% Professorship code is the same in all languages
\newcommand{\PROFCODE}{TU-53}
\newcommand{\KEYWORDS}{presales, sales process, ...}
\newcommand{\FKEYWORDS}{myyntiprosessi}

\newcommand{\LANGUAGE}{English}
\newcommand{\FLANGUAGE}{Englanti}


% Author is the same for all languages
\newcommand{\AUTHOR}{Nora Huovila}


% Currently the English versions are used for the PDF file metadata
% Set the PDF title
\hypersetup{pdftitle={\TITLE\ \SUBTITLE}}
% Set the PDF author
\hypersetup{pdfauthor={\AUTHOR}}
% Set the PDF keywords
\hypersetup{pdfkeywords={\KEYWORDS}}
% Set the PDF subject
\hypersetup{pdfsubject={Master's Thesis}}


% Layout settings
% ------------------------------------------------------------------

% When you write in English, you should use the standard LaTeX
% paragraph formatting: paragraphs are indented, and there is no
% space between paragraphs.
% When writing in Finnish, we often use no indentation in the
% beginning of the paragraph, and there is some space between the
% paragraphs.

% If you write your thesis Finnish, uncomment these lines; if
% you write in English, leave these lines commented!
% \setlength{\parindent}{0pt}
% \setlength{\parskip}{1ex}

% Use this to control how much space there is between each line of text.
% 1 is normal (no extra space), 1.3 is about one-half more space, and
% 1.6 is about double line spacing.
% \linespread{1} % This is the default
% \linespread{1.3}

% Bibliography style
% acm style gives you a basic reference style. It works only with numbered
% references.
\bibliographystyle{plainnat}
% Plainnat is a plain style that works with both numbered and name citations.
% \bibliographystyle{plainnat}


% Extra hyphenation settings
% ------------------------------------------------------------------
% You can list here all the files that are not hyphenated correctly.
% You can provide many \hyphenation commands and/or separate each word
% with a space inside a single command. Put hyphens in the places where
% a word can be hyphenated.
% Note that (by default) LaTeX will not hyphenate words that already
% have a hyphen in them (for example, if you write ``structure-modification
% operation'', the word structure-modification will never be hyphenated).
% You need a special package to hyphenate those words.
\hyphenation{di-gi-taa-li-sta yksi-suun-tai-sta}



% The preamble ends here, and the document begins.
% Place all formatting commands and such before this line.
% ------------------------------------------------------------------
\begin{document}
% This command adds a PDF bookmark to the cover page. You may leave
% it out if you don't like it...
\pdfbookmark[0]{Cover page}{bookmark.0.cover}
% This command is defined in aalto-thesis.sty. It controls the page
% numbering based on whether the doublenumbering option is specified
\startcoverpage

% Cover page
% ------------------------------------------------------------------
% Options: finnish, english, and swedish
% These control in which language the cover-page information is shown
\coverpage{english}


% Abstracts
% ------------------------------------------------------------------
% Include an abstract in the language that the thesis is written in,
% and if your native language is Finnish or Swedish, one in that language.

% Abstract in English
% ------------------------------------------------------------------
\thesisabstract{english}{
Abstract in English

% Abstract in Finnish
% ------------------------------------------------------------------
\thesisabstract{finnish}{
Suomenkielinen abstrakti



% Acknowledgements
% ------------------------------------------------------------------
% Select the language you use in your acknowledgements
\selectlanguage{english}

% Uncomment this line if you wish acknoledgements to appear in the
% table of contents
%\addcontentsline{toc}{chapter}{Acknowledgements}

% The star means that the chapter isn't numbered and does not
% show up in the TOC
\chapter*{Acknowledgements}

tähän suuret kiitokset kaikille !
\vskip 10mm

\noindent Helsinki, \DATE
\vskip 5mm
\noindent\AUTHOR

% Acronyms
% ------------------------------------------------------------------
% Use \cleardoublepage so that IF two-sided printing is used
% (which is not often for masters theses), then the pages will still
% start correctly on the right-hand side.
%\cleardoublepage
% Example acronyms are placed in a separate file, acronyms.tex
% \input{acronyms}

%\addcontentsline{toc}{chapter}{Abbreviations and Acronyms}
%\chapter*{Abbreviations and Acronyms}

% The longtable environment should break the table properly to multiple pages,
% if needed

%\noindent
%\begin{longtable}{@{}p{0.25\textwidth}p{0.7\textwidth}@{}}
%2k/4k/8k mode & COFDM operation modes \\
%3GPP & 3rd Generation Partnership Project \\
%ESP & Encapsulating Security Payload; An IPsec security protocol \\
%FLUTE  & The File Delivery over Unidirectional Transport protocol \\
%e.g.& for example (do not list here this kind of common acronymbs or abbreviations, but only those that are %essential for understanding the content of your thesis. \\
%note & Note also, that this list is not compulsory, and should be omitted if you have only few abbreviations

%\end{longtable}


% Table of contents
% ------------------------------------------------------------------
\cleardoublepage
% This command adds a PDF bookmark that links to the contents.
% You can use \addcontentsline{} as well, but that also adds contents
% entry to the table of contents, which is kind of redundant.
% The text ``Contents'' is shown in the PDF bookmark.
\pdfbookmark[0]{Contents}{bookmark.0.contents}
\tableofcontents

% List of tables
% ------------------------------------------------------------------
% You only need a list of tables for your thesis if you have very
% many tables. If you do, uncomment the following two lines.
% \cleardoublepage
% \listoftables

% Table of figures
% ------------------------------------------------------------------
% You only need a list of figures for your thesis if you have very
% many figures. If you do, uncomment the following two lines.
% \cleardoublepage
% \listoffigures

% The following label is used for counting the prelude pages
\label{pages-prelude}
\cleardoublepage

%%%%%%%%%%%%%%%%% The main content starts here %%%%%%%%%%%%%%%%%%%%%
% ------------------------------------------------------------------
% This command is defined in aalto-thesis.sty. It controls the page
% numbering based on whether the doublenumbering option is specified
\startfirstchapter

% Add headings to pages (the chapter title is shown)
\pagestyle{headings}

% The contents of the thesis are separated to their own files.
% Edit the content in these files, rename them as necessary.
% ------------------------------------------------------------------

% \input{1introduction.tex}

\chapter{Introduction}
\label{chapter:intro}






\section{Background}

hieman taustaa tutkimukselle ja caselle

\section{Research problem and questions}
\label{section:constructive}

What is presales? How is it defined in relation to the sales process and to the buying process?

How presales should be organized at Sievo. 

\section{Research framework and objectives}
\label{section:structure}

The objective is to imporve existing practices in the case company. therefore I chose the constructive case study as a reserach method. It focuses on solving business problems in real-life situations of organisations. Kasanen 1993. Depending on the nature of the case study existing literature should be used as a basis of analysis. Yin, 2003. 

Qualitative = purpose to understand human behaviour and reasons behind it by using interviews and observations. 

\section{Scope of the study}
\label{section:structure}

\section{Terminology}
\label{section:Terminology}


\noindent
\begin{longtable}{@{}p{0.25\textwidth}p{0.7\textwidth}@{}}
B2B & Business to Business \\
RFP & Request for Proposal - Either formal document or informal request sent by buyer to potential suppliers for the purpose of collecting proposals a certain business need or problem \\
RFI & Request for Information - A formal request from a customer requesting information on supplier capabilities to be used in possily upcoming request for proposal \\
Proposal  & A written offer from a seller to a prospective buyer. In the literature, also terms bid and offer are used to depict the same thing \\
ERP & \\

\end{longtable}

% \input{2background.tex}

\chapter{Case Sievo}
\label{chapter:Case Sievo}

Tässä osiossa esittelen siveon yrityksenä myyntiorganisaation nykyistä toimintaan ja toivottuja tuloksia tutkimukselta.

\section{Sievo as a company}



sotfware product company
sophisticated procurement performance management solutions enabling comprehensive visibility on corporate spending, measurement of procurement contribution to bottom line and track contract compliance. 

Data is gathered from the cutomers erp system and the measurement is build on top of that 


\section{Sales function}

Kuvailen myynnin nykytilaa, miten se on organisoitunut. Mitkä ovat sen tämän hetkiset haasteet
Maybe joku organisaatio kaavio. 


\section{Sales process 'as is'}

Mikä on myyntiprosessin nykytila?

\section{Presales}

Mikä on presales tämän hetkinen rooli. Miten se toimii suhteessa muuhun firmaan

\section{Internal expectation for the study}

mitä dipaltani toivotaan/odotetaan

% \input{3environment.tex}

\chapter{Literature review}
\label{chapter:Theory}


In this theory part of my thesis I will start by going through the theory behind the development of sales processes. I wil start by introducing the general theories introduced in the 80's and narrow in down to the procesees of high technology selling today. 
In the second part of the theory I will examine the same process from the buyer's side. 
Finally I will explain the concept of service development on which I will base my recommendation of the organisation of the presales. 


\section{Sales process}
\label{sec:environments}
A process is a set of set of actions done in a sequence that aim for a certain outcome. A sales process is a set of actions that target to reach a sale. In this section I go throught the trends that have influeced the evolution of the sales process. I will lead this subject towards the specific theme of high technology sales proces.  


\subsection{Seven steps of personal selling}

The seven steps of personal selling is one of the most oldest and most referred to processes in the field of sales. It used to be referred to as the most expensive and exclusive from of marketing. 

The seven steps were identified already earlier but Dubinsky's research in 1980, the steps reached their most referred form. Before there were multiple version used. 

Dubinsky studied the underlying dimensions that characterize the techniques salespeople use during the execution of those steps. The main emphasis was on how the salespeople themselves see the importance of the various techniques. Dubinsky distinguished the most important steps and describe the sales process based on those. 


\newline \newline \emph{Prospecting}\newline 
Prospecting is the first phase of the process where a salesperson is searching for new potential customers. The purpose is to find customers who have not yet purchased from the company. This phase also usually includes qualifying the prospect as a potential one. \citep{Jobber}\newline \newline
\emph{Preapproach}\newline 
Preapproaching phase is the investigation phase before the actual visit. It includes the salesperson getting to know the customers need, industry, seeing for potential past correspondance and preparing material for the actual meeting. \citep{Moncrief}\newline \newline
\emph{Approach}\newline
The approach could also be referred as the first impression. It is the first moments with the customer when the trust and impression are made. Dubinsky mentions that at this stage several different types of approaches might be taken for example the product approach, consumer approach, the consultative approach. \citep{Dubinsky} \newline \newline
\emph{Presentation}\newline
The presentation is the main part of the meeting and require an adequate amount of preparations. This includes sharing the information regarding the product and demonstrations. It could be one meeting or occur in several instances. \citep{Moncrief}\newline \newline
\emph{Overcoming objections}\newline
The customer might have doubts and hesitations regarding the purchase. At this stage the salesperson tries to get to the bottom of the concerns and overcome them. \citep{Moncrief}\newline \newline
\emph{Close}\newline
Close is the buy or order of the product. The objections have been overcome at this stage and the commitment to buy has been achieved \citep{Moncrief} \newline \newline
\emph{Follow-up}\newline
The follow-up step, also referred as post-sale includes the assumption that the sales is not over with the order. After the order the salesperson makes sure that everything is operating smoothly and the customer is happy. \citep{Dubinsky}
\newline 
\begin{figure}[ht]
  \begin{center}
    \includegraphics[width=\textwidth]{sevensteps.pdf}
    \caption{The seven steps of personal selling}
    \label{fig:seven}
  \end{center}
\end{figure}
\subsection{Customer Centric approach}

The concept of putting the customer first is not new but it has definitely risen its importance during the past decade. \citp{Bolton} ~\citep{Shah}. Companies have made the shift from production oriented towards customer oriented. The focus used to be on optimizing the internal processes and reaching the economies of scale, but now the importance of the customer is aknowledged. 

The heart of costumer centricity lies if creating a win win situation; selling products and delivering value to the customer. \citep{Shah} Keeping the customer at the core of all the processes also affects the companies financial performance. Studies have shown that companies measuring customer satidfaction, loyalty and quality preceived also do better financially. The emphasis is on valuing the relationship with the customer. 

The idea in customer centricity is not only set the sales to better serve the customer but the direct the whole organisztion to serve the customer better. This also means alingning the product development, implementation, hr and other functions towards the customer's need and interests. The whole organisation should be on the mission of serving the customer better. It is not enough anymore that the sales act as an interface towards the customer. 
\newline

The customer centricity has modified the seven steps of personal selling according to Moncrief et al. as the following. These modification also take in account the other impacting trends such as improved technology and increased knowledge on the buyer's side. \newline \newline \emph{Prospecting - Customer retention and deletion}\newline 
The prospecting itself might even be outsourced since the main emphasis is put on the most important customer and retaining the relationship with them \citep{Moncrief}\newline \newline
\emph{Preapproach - Database and knowledge management}\newline 
Big share of the marketing and selling activities revolve around the maintenance of the knowledge base. Salespeople keep the customer database updated which serves as a comprehensive source of knowledge. \citep{Moncrief}\newline \newline
\emph{Approach - Nurturing the relationship}\newline
Since nowadays much effort is put on building the relationship with the customer the selling activites have also been directed towards that. Relationship maintenance is a long term process. In the traditional approach this step was about making the first impression and in an updated version it serves more for the activities to build the foundation for the relationship. \citep{Moncrief} \newline \newline
\emph{Presentation - Marketing the product}\newline
Instead of presenting the products the shift has been towards gathering more information from the customer and listening to the needs and problems on their side. Those requirements are then mirrored against the product offerings. The role of the salesperson is to conduct information between the buyer and seller organization.   \citep{Moncrief}\newline \newline
\emph{Overcoming objections - Problem solving}\newline
The focus has shifted from overcoming objections to acting as a consultant finding effective solutions to the identified problems. The salesperson should be seen as a valued partner that bring value to the customer's end. \citep{Moncrief}\newline \newline
\emph{Close - Adding value / satisfying needs}\newline
The customer and the seller are seen are working together towards a common goal. The ultimate goal for the salesperson is to achieve a strong relationship and a partnership towards instead of just closing a deal. This means continuous business with the client. \citep{Moncrief} \newline \newline
\emph{Follow-up - Customer relationship maintenance}\newline
The maintenance means that the selling organization has assigned a person or a team to keep the relationship alive. The reward systems of the salespeople have also been altered towards long term relationship building.  \citep{Moncrief}
\begin{figure}[ht]
  \begin{center}
    \includegraphics[width=\textwidth]{cosseven.pdf}
    \caption{The process evolving with the customer centric approach}
    \label{fig:cosseven}
  \end{center}
\end{figure}

\subsection{High Tech selling: solution selling}


Compared to the costumer centric selling introduced in the earlier chapter, in solution selling  the focus is on the fifth and sixth step of the process; the problem solving and adding value. The whole approach is built around those steps. One could think that the costumer centric approach is a bigger theme that has been further narrowed down to solution selling fopcusing on the problem

According to Eades (2006) being solution centric means to define the organization by the problems it solves for the customer. All the functions within the organization are aligned to tackle the customer's problem. Its value is measured not only on the financial performance but also on the outcomes that their customer receives. 

High technology products are typically implemented to solve a certain customer problem. According to Sharma et al. (2008) these are however typically sold more with product orientation than  customer orientation. In the field of technology there are several indications that selling technology should acquire a more solution centric approach.

The high complexity of the technology services and products make their purchasing also complex. Usually the procurement acquires thorough capital and technical considerations. This also sets higher expectations to the sales side. The salesperson is seen more as a consultant solving the problem which sets high standards for the sales people. \citep{Sharma}

To match the expectations the sales should be seen as teams. Sales should be organized around tackling the customer's problem and usually a single salesperson is not enough. The selling in typically done in teams with multidiscliplinary functions to answer the customer concern thoroughly. The sales needs to have the knowledge regarding the technical specifications and the fulfillment. They should be able to maintain technical dialog and offer technical support when needed.  \citep{Sharma}

The people skills required from the sales side also have a new meaning. Manage and influence the internal stakeholders and align those to solve the customer problem. The sales should be able to orchestrate the internal resources to satisfy the clients needs. The sales acts are the main gate of information between the customer and the rest of the organization. Sales is also the main source of information towards the products development.  \citep{Sharma}

The implication of the solution selling to the sales process is the emphasis on the problem identification, presentation of the solution and continued costumer support. The process according to \citep{Eades}. The opportunity planning is no longer seen as part of the sales process, instead as a separate entity that come before the sales execution. The process now begins from creating the opportunity inside the customer's organization. \newline \newline\emph{Create}\newline 
The purpose of the first step is to create interest inside the customer organization and identify a potential sponsor for the purchase. During this phase also an opportunity and competitive strategy should be assessed. \citep{Eades}\newline \newline
\emph{Qualify}\newline 
In the qualify phase the main pain is identified and a vision is created and communicated to solve the problem. Simultaneously one should also negotiate the access to power - to a person who can make the actual decisions. 
 \citep{Eades}\newline \newline
\emph{Develop}\newline
When the person with the power has been achieve, also his pain should be identified. Then an agreement should be made to explore further. During this phase also evaluation criteria should be identified. 
\citep{Eades} \newline \newline
\emph{Prove}\newline
Here the preliminary solutions should be presented and proposal for the implementation. During this phase a verbal agreement should be achieved before moving on to the next step.  \citep{Eades}\newline \newline
\emph{Negotiate}\newline
This step serves to prepare for the signing of the contracts. Final agreements should be reached. \citep{Eades}\newline \newline
\emph{Close}\newline
The purpose of the close phase is to get the documents signed and the official approval of the project. \citep{Eades} \newline \newline
\emph{Implement}\newline
In the final phase the solution is implemented to the customer and satisfaction is measured. The identification of new potential opportunities also begins and achiving the reference story \citep{Eades}
\newline


\begin{figure}[ht]
  \begin{center}
    \includegraphics[width=\textwidth]{kolmas.pdf}
    \caption{Evolution of the steps towards high tech sales}
    \label{fig:kolmas}
  \end{center}
\end{figure}

\subsection{From solution selling to insight selling}
The future trends in sales processes are yet to be defined. In 2012 Adamson et al. wrote and interesting article on the Harvard Business review on new themes of selling. They were stating that solution selling is coming to its end. 

The situation is shifting towards purchasers being ahead of the salespeople. They do not need the sales people anymore to consult them on what they need. Instead the clients have a deep understanding of their problems and a well defined RFP document to find the best supplier to fill their needs. The problems are already diagnosed and the procurement organization has taken over the search for the best supplier. The sales conversations are turning into fulfillment conversations. 
The key in insight selling is to identify the potential customers that are agile enough and ready for the change. What is now needed from the sales people is the ability to challenge the buyer organization radically. Give them new insight and reveal them problems they do not know they have. 
This also includes coaching the buyer to get the deal done. The salesperson usually has more experience on how to influence people inside the organizations. 

According to Adamson et al. organizations need to fundamentally rethink the training and support provided to their sales. 

\section{Organization buying behaviour theory}


Fundamentally defining the sales begins with understanding the buying process. In this chapter I will introduce the theories behind the buying. The organizational buying behavior theory introduces three types of buying, the recurring purchase, the modified recurring purchase and the completely new purchase. I will concentrate on my thesis only the modified and completely new theory. 

different types of buying
\citep{Lewin}
several 

\subsection{Buygrid framework}

developed by Robinson, Faris, Wind 1967

\subsection{MERPAP framework}
Research around software and IT buying is scarce.  The study made by Jacques Verville and Allannah Halingten is one of the few. They developed a model to describe the buying process of an ERP software. Although the Sievo software is not an ERP system similar elements can be found in the dimensions that are defined during the purchasing phase such as the benefits, riks, challenges, costs etc. 

According to Verville and Halingten some parts of the existing models of the organisation buying behaviour can be adapted in buying IT while others may not. Some assumption that do not match with the IT procurement is that
- they overly emphasize the influences of the buying unit while neglecting the dynamics of the process involved. 

Based on the multiple case studies of the process of acquiring an ERP software Verville and Halingten identified six distinct common phases. These steps are not linear and can be conducted concurrently and iteratively. 

The planning phase comprises of the team formation with defining the roles, needed knowledge and skills leaders and whether for example external consultants are needed. The acquisition strategy is also defined. What are the technical requirements based on the existing solutions, functional requirements, the different users and existing processes. The evaluation criteria for the vendors is also defined and a market analysis is made. The different deliverables for the different phases are also specified. 

The planning phase requires the majority of the time spend during the process. The planning ends with a RFP document sent to the long list of vendor acquired on the market analysis. 

After the delivering of the RFP documents the short list is made of the vendor that most likely meet the needs of the organization. Information is gathered throughout the process. It starts with the higher level of information and as the process goes further the evaluation criteria is based on a more detailed information.

A recommendation of the preferred supplier is usually presented to a outside group; board of directors or steering committee for instance. The official choice is usually done by the steering group.

The negotiations comprise of two parts the legal and the business negotiations.  The business negotiations happen concurrently with the other phases; during the planning, information search, selection and evaluation activities. the legal negoations only begin with the selected supplier. \citep{Verville}


\subsection{roles of buying}

\subsection{roles in system buying}

enemmän rooleja mukana ostamisessa kuin tavallisessa. lattral extent on suurempi samoin vertical extent. 

The legal negoatiation occur after the choice of the vendor. The business part of the negotiations is doe during the process. Once achieved the consensus the implementation of the software may begin. 

% Comment: If your sentence ends in a capital letter, like here, you should
% write \@ before the period; otherwise LaTeX will assume that this is not
% really an end of the sentence and will not put a large enough space after the
% period. That is, LaTeX assumes that you are (for example), enumerating using
% capital roman numerals, like I. do something, II. do something else. In this
% case, the periods do not end the sentence.

% Similarly, if you do need a normal space after a period (instead of
% the longer sentence separator), use \  (backslash and space) after the
% period. Like so: a.\ first item, b.\ second item.



% These definitions are only used in the example images; you will not
% need them for your thesis...

\section{Service concept}

% \input{4methods.tex}

\chapter{Empirical part}
\label{chapter:methods}
\section{Methods}

internal interviews
external interviews

\section{Results}
% An example of a traditional LaTeX table
% ------------------------------------------------------------------
% A note on underfull/overfull table cells and tables:
% ------------------------------------------------------------------
% In professional typography, the width of the text in a page is always a lot
% less than the width of the page. If you are accustomed to the (too wide) text
% areas used in Microsoft Word's standard documents, the width of the text in
% this thesis layout may suprise you. However, text in a book needs wide
% margins. Narrow text is easier to read and looks nicer. Longer lines are
% hard to read, because the start of the next line is harder to locate when
% moving from line to the next.
% However, tables that are in the middle of the text often would require a wider
% area. By default, LaTeX will complain if you create too wide tables with
% ``overfull'' error messages, and the table will not be positioned properly
% (not centered). If at all possible, try to make the table narrow enough so
% that it fits to the same space as the text (total width = \textwidth).
% If you do need more space, you can either
% 1) ignore the LaTeX warnings
% 2) use the textpos-package to manually position the table (read the package
%    documentation)
% 3) if you have the table as a PDF document (of correct size, A4), you can use
%    the pdfpages package to include the page. This overrides the margin
%    settings for this page and LaTeX will not complain.
% ------------------------------------------------------------------
% Another note:
% ------------------------------------------------------------------
% If your table fits to \textwidth, but the cells are so narrow that the text
% in p{..}-formatted cells does not flow nicely (you get underfull warnings
% because LaTeX tries to justify the text in the cells) you can manually set
% the text to unjustified by using the \raggedright command for each cell
% that you do not want to be justified (see the example below). \raggedleft
% is also possible, of course...
% ------------------------------------------------------------------
% If you need to have linefeeds (\\) inside a cell, you must create a new
% paragraph-formatting environment inside the cell. Most common ones are
% the minipage-environment and the \parbox command (see LaTeX documentation
% for details; or just google for ``LaTeX minipage'' and ``LaTeX parbox'').
%\begin{table}
%\begin{tabular}{|p{2cm}|p{3.8cm}|p{4.5cm}|p{1.1cm}|}
% Alignment of sells: l=left, c=center, r=right.
% If you want wrapping lines, use p{width} exact cell widths.
% If you want vertical lines between columns, write | above between the letters
% Horizontal lines are generated with the \hline command:
%\hline % The line on top of the table
%\textbf{Code} & \textbf{Name} & \textbf{Methods} & \textbf{Area} \\
%\hline
% Place a & between the columns
% In the end of the line, use two backslashes \\ to break the line,
% then place a \hline to make a horizontal line below the row
%T-110.6130 & Systems Engineering for Data Communications
   % Software & \raggedright Computer simulations, mathematical modeling,
  %experimental research, data analysis, and network service business
 % research methods, (agile method) & T-110 \\
%\hline
%\multicolumn{2}{|p{6.25cm}|}{Mat-2.3170 Simulation (here is an example of
 %multicolumn for tables)}& Details of how to build simulations & T-110 \\
% The multicolumn command takes the following 3 arguments:
% the number of cells to merge, the cell formatting for the new cell, and the
% contents of the cell
%\hline
%S-38.3184 & Network Traffic Measurements and Analysis
%& \raggedright How to measure and analyse network
 % traffic & T-110 \\ \hline
%\end{tabular} % for really simple tables, you can just use tabular
% You can place the caption either below (like here) or above the table
%\caption{Research methodology courses}
% Place the label just after the caption to make the link work
%\label{table:courses}
%\end{table} % table makes a floating object with a title



% \input{5implementation.tex}

\chapter{Recommendations}
\label{chapter:Recommendations}



% \input{6evaluation.tex}

\chapter{Evaluation/Discussion/Conclusion}
\label{chapter:evaluation}



% \input{8conclusions.tex}




% Load the bibliographic references
% ------------------------------------------------------------------
% You can use several .bib files:
% \bibliography{thesis_sources,ietf_sources}
\bibliography{ref}


% Appendices go here
% ------------------------------------------------------------------
% If you do not have appendices, comment out the following lines
\appendix
% \input{appendices.tex}

\chapter{First appendix}
\label{chapter:first-appendix}

This is the first appendix. You could put some test images or verbose data in an
appendix, if there is too much data to fit in the actual text nicely.

For now, the Aalto logo variants are shown in Figure~\ref{fig:aaltologo}.

\begin{figure}
\begin{center}
\subfigure[In English]{\includegraphics[width=.8\textwidth]{aalto-logo-en}}
\subfigure[Suomeksi]{\includegraphics[width=.8\textwidth]{aalto-logo-fi}}
\subfigure[Pä svenska]{\includegraphics[width=.8\textwidth]{aalto-logo-se}}
\caption{Aalto logo variants}
\label{fig:aaltologo}
\end{center}
\end{figure}


% End of document!
% ------------------------------------------------------------------
% The LastPage package automatically places a label on the last page.
% That works better than placing a label here manually, because the
% label might not go to the actual last page, if LaTeX needs to place
% floats (that is, figures, tables, and such) to the end of the
% document.
\end{document}
